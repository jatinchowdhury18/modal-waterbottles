% !TeX program = pdflatex
% !BIB program = bibtex
% Template LaTeX file for DAFx-19 papers

%------------------------------------------------------------------------------------------
%  !  !  !  !  !  !  !  !  !  !  !  ! user defined variables  !  !  !  !  !  !  !  !  !  !  !  !  !  !
% Please use these commands to define title and author(s) of the paper:
\def\papertitle{Water bottle Synthesis with Modal Signal Processing}
\def\paperauthorA{ Jatin Chowdhury, Elliot K. Canfield-Dafilou, and Mark Rau}

% Authors' affiliations have to be set below

%------------------------------------------------------------------------------------------
\documentclass[twoside,a4paper]{article}
\usepackage{etoolbox}
\usepackage{dafx_20}
\usepackage{amsmath,amssymb,amsfonts,amsthm}
\usepackage{euscript}
\usepackage[latin1]{inputenc}
\usepackage[T1]{fontenc}
\usepackage{ifpdf}

\usepackage[english]{babel}
\usepackage{caption}
\usepackage{subfig} % or can use subcaption package
\usepackage{xcolor}
\usepackage{soul}

\input glyphtounicode
\pdfgentounicode=1

\setcounter{page}{1}
\ninept

% build the list of authors and set the flag \multipleauth to handle the et al. in the copyright note (in DAFx_20.sty)
%==============================DO NOT MODIFY =======================================
\newcounter{numauth}\setcounter{numauth}{1}
\newcounter{listcnt}\setcounter{listcnt}{1}
\newcommand\authcnt[1]{\ifdefined#1 \stepcounter{numauth} \fi}

\newcommand\addauth[1]{
\ifdefined#1 
\stepcounter{listcnt}
\ifnum \value{listcnt}<\value{numauth}
\appto\authorslist{, #1}
\else
\appto\authorslist{~and~#1}
\fi
\fi}
%======DO NOT MODIFY UNLESS YOUR PAPER HAS MORE THAN 10 AUTHORS========================
%==we count the authors defined at the beginning of the file (paperauthorA is mandatory and already accounted for)
\authcnt{\paperauthorB}
\authcnt{\paperauthorC}
\authcnt{\paperauthorD}
\authcnt{\paperauthorE}
\authcnt{\paperauthorF}
\authcnt{\paperauthorG}
\authcnt{\paperauthorH}
\authcnt{\paperauthorI}
\authcnt{\paperauthorJ}
%==we create a list of authors for pdf tagging, for example: paperauthorA, paperauthorB, ... and paperauthorF (last author)
\def\authorslist{\paperauthorA}
\addauth{\paperauthorB}
\addauth{\paperauthorC}
\addauth{\paperauthorD}
\addauth{\paperauthorE}
\addauth{\paperauthorF}
\addauth{\paperauthorG}
\addauth{\paperauthorH}
\addauth{\paperauthorI}
\addauth{\paperauthorJ}
%====================================================================================

\usepackage{times}
% Saves a lot of ouptut space in PDF... after conversion with the distiller
% Delete if you cannot get PS fonts working on your system.

% pdf-tex settings: detect automatically if run by latex or pdflatex
\newif\ifpdf
\ifx\pdfoutput\relax
\else
   \ifcase\pdfoutput
      \pdffalse
   \else
      \pdftrue
\fi

\ifpdf % compiling with pdflatex
  \usepackage[pdftex,
    pdftitle={\papertitle},
    pdfauthor={\paperauthorA},
    colorlinks=false, % links are activated as colror boxes instead of color text
    bookmarksnumbered, % use section numbers with bookmarks
    pdfstartview=XYZ % start with zoom=100% instead of full screen; especially useful if working with a big screen :-)
  ]{hyperref}
  \pdfcompresslevel=9
  \usepackage[pdftex]{graphicx}
%   \usepackage[figure,table]{hypcap}
\else % compiling with latex
  \usepackage[dvips]{epsfig,graphicx}
  \usepackage[dvips,
    colorlinks=false, % no color links
    bookmarksnumbered, % use section numbers with bookmarks
    pdfstartview=XYZ % start with zoom=100% instead of full screen
  ]{hyperref}
  % hyperrefs are active in the pdf file after conversion
  % \usepackage[figure,table]{hypcap}
\fi
\usepackage[hypcap=true]{caption}

% My packages
\usepackage{tikz}
\usepackage{tkz-euclide}
\usetkzobj{all}
\usepackage{cleveref}

\usepackage{listings}
\definecolor{codegreen}{rgb}{0,0.6,0}
\definecolor{codegray}{rgb}{0.5,0.5,0.5}
\definecolor{codepurple}{rgb}{0.58,0,0.82}
\definecolor{backcolour}{rgb}{0.95,0.95,0.92}
 
\lstdefinestyle{mystyle}{
    backgroundcolor=\color{backcolour},   
    commentstyle=\color{codegreen},
    keywordstyle=\color{magenta},
    numberstyle=\tiny\color{codegray},
    stringstyle=\color{codepurple},
    basicstyle=\footnotesize,
    columns=flexible,
    breakatwhitespace=false,         
    breaklines=true,                 
    captionpos=b,                    
    keepspaces=true,                               
    showspaces=false,                
    showstringspaces=false,
    showtabs=false,                  
    tabsize=4
}
 
\lstset{style=mystyle}

\DeclareMathAlphabet{\mathpzc}{OT1}{pzc}{m}{it}
\newcommand{\z}{\mathpzc{z}}

\title{\papertitle}

\affiliation{
\paperauthorA \, }
{\href{http://ccrma.stanford.edu}{Center for Computer Research in Music and Acoustics} \\ Stanford University \\ Palo Alto, CA \\ {\tt\{jatin|kermit|mrau\}@ccrma.stanford.edu}}

\begin{document}
% more pdf-tex settings:
\ifpdf % used graphic file format for pdflatex
  \DeclareGraphicsExtensions{.png,.jpg,.pdf}
\else  % used graphic file format for latex
  \DeclareGraphicsExtensions{.eps}
\fi

\maketitle
%
\begin{abstract}
We present a method for accurately synthesizing the acoustic response
of a water bottle using modal signal processing. We take extensive
measurements of two water bottles with considerations for how the amount of water within the bottles and stickers attached to the exterior
of the bottles affect the sound. We perform modal analysis of these measurements and implement a modal water bottle model as a real-time synthesizer.
\end{abstract}

\section{Introduction} \label{sec:intro}
%
Previous works have examined the use of modal signal processing
for synthesizing carillon bells
\cite{canfielddafilou:werner:bellEffects:2017,rau:das:canfielddafilou:carillon:2019},
artificial reverberation \cite{abel2014a}, cymbal synthesis \cite{travis_cymbals},
and more \cite{abel_kurt_modal}.
In this paper, we extend this previous work to use modal synthesis
for the accurate modelling of water bottle acoustics.

Although most water bottles are not designed to function primarily as musical
instruments, the authors have noticed that certain water bottles
can produce a pleasing resonant sound when struck with a knee, hand,
or other body part. The authors further noticed that water bottles can
produce a great variety of sounds, depending not only on the shape and
material of the bottle, but also on the amount of liquid contained
within the bottle, as well as the potential placement of stickers on the
exterior of the bottle. Water bottle acoustics have not gone unnoticed
by water bottle manufacturers, as at least one prominent manufacturer claims
to be well aware of the pleasing acoustic properties of their bottles
\cite{hydroflask_email}.

Other things to cite: 
% https://research.cs.cornell.edu/HarmonicShells/HarmonicShells09.pdf
% https://www.springer.com/gp/book/9783642164071
% we should probably cite helmholtz 
\cite{russelBEER} modes of beer bottle
% the following two citations are nicely described here: http://www.kilty.com/coffee.htm
\cite{french1983vino} wine glass pitch
\cite{chen2005does} wine glass pitch
\cite{rossing1990wine} comparison of wine glasses to bells
\cite{bragg1921world} pitch change noticed in beer glass
\cite{crawford1982hot} pitch change in hot chocolate
\cite{morrison2002sound} modes of coffee cup change with bubbles in instant coffee % https://acoustics.org/pressroom/httpdocs/143rd/Rossing.html
\cite{morrison2014acoustics} acoustic concepts demonstrated with coffee cup
\cite{thines2012wine} pitch and sound of shattering wine glass
\cite{piacsek2015glass} mass loading the resonant structure of a wine glass with liquid
\cite{jundt2006vibrational} vibrational modes of wine glass with different levels of liquid
\cite{courtois2008tuning} how the liquid is distributed in the glass matters (this is out in for the maple syrup)
\cite{apfel1982whispering} exploration of circular capillary waves on the surface of the liquid when exciting wine glass with wet finger
\cite{arbel2017wine} for some reason these people coupled a vibrating string to a wine glass to make it resonate sympathetically... really not sure why :D


%  hydroflask is made of high quality food grade 18/8 stainless steel, but they're vacuum insulated and with "custom powdered coat"
% some materials properties: https://www.theworldmaterial.com/what-is-18-8-stainless-steel/

In this writing, we take measurements from a 32 oz. Wide Mouth
HydroFlask\footnote{\url{https://www.hydroflask.com/32-oz-wide-mouth/}},
compared to the water bottle given to attendees of the 2019 DAFx
conference (seen in Fig.~\ref{fig:dafx_measure}), measured containing different amounts of water and with different placements of stickers on the exteriors of the
bottles.

The structure of the paper is as follows: In \S\ref{sec:measure} we describe
our water bottle acoustical measurement procedure. \S\ref{sec:analysis}
contains modal analysis of the water bottle measurement data.
Finally in \S\ref{sec:results} we discuss our results, and the
implementation of a full water bottle synthesizer.

\section{Measurements} \label{sec:measure}
%
In this study, we want to study the vibrational modes of the water bottle independently from the method it is struck. To do this, we strike the water bottle with a force hammer and measure the surface velocity of the water bottle at a corresponding point using a laser Doppler vibrometer. We additionally captured the near field radiation using a pressure microphone. From both these measurements, we deconvolve out of the signal the impact of the force hammer. The full setup can be seen in \cref{fig:dafx_measure}.

We made measurements of both a 32 oz. Wide mouth Hydroflask and the complimentary DAFx19 water bottles no water as well as $1/32$, $1/16$, $1/8$, $1/4$, $1/2$, and full with water. We additionally made measurements of the Hydroflask with the exterior covered in vinyl stickers as well as several intermediate and differently placed amounts of stickers. 

Finally, we also wanted some understanding of different impacts on the water bottles. To make these measurements, we adhered an accelerometer to the inside of the Hydroflask and struck the outside (at the location of the accelerometer) with a variety of drum mallets and body parts.  



%
% \begin{figure}[!htb]
%     \centering
%     \includegraphics[width=3in]{Figures/hydroflask_measure}
%     \caption{\it{Measurement setup for the HydroFlask water bottle}}
%     \label{fig:hydro_measure}
% \end{figure}
%
\begin{figure}[!htb]
    \centering
    \includegraphics[width=3in]{Figures/dafx_measure}
    \caption{\it{Measurement setup for the DAFx water bottle}}
    \label{fig:dafx_measure}
\end{figure}

\section{Analysis} \label{sec:analysis}
%
\subsection{Modal Analysis} \label{sec:modal-analysis}
%
Similar to the carillon bells modelled in \cite{canfielddafilou:werner:bellEffects:2017,rau:das:canfielddafilou:carillon:2019},
we can use modal analysis to model the water bottle sounds as
a sum of exponentially decaying sinusoids.
\begin{equation}
    y(t) = \sum_{m=1}^M \alpha_m e^{j\omega_m t} e^{-t/\tau_m}
    \label{eq:modal-def}
\end{equation}
%
where $\alpha_m$ is the complex amplitude, $\omega_m$ is the mode
frequency, and $\tau_m$ is the decay rate for each mode $m$.
\newline\newline
For modal analysis we use the helper functions
provided by the \texttt{Python} audio signal processing library
\texttt{audio\_dspy}\footnote{\url{https://github.com/jatinchowdhury18/audio_dspy}}.
This process involves the following steps:
\begin{enumerate}
    \item Picking the modal frequencies from the original recording.
    \item Estimate the decay rate of each mode.
    \item Estimate the complex amplitude of each mode.
\end{enumerate}
%
The process is shown in  full in \cref{fig:modal_analysis}.
\newline\newline
For finding the mode frequencies, we use a simple peak picking
algorithm over the Fourier Transform of the original signal.
\newline\newline
For estimating the mode decay rates, we begin by filtering
the signal using a 4th order Butterworth bandpass filter
centered on the mode frequency, with a bandwidth of 30 Hz.
We then apply a Root-Mean-Squared level detector as defined
in \cite{giannoulis2012compressor} to estimate the energy
envelope of the mode. Finally, we use a linear regression
to estimate the slope of the energy envelope (measured in
Decibels).
\newline\newline
After computing the mode frequencies and decay
rate, we can do a simple least squares fit to
estimate the complex amplitude of each mode that
most accurately resynthesizes the original recording.
This process is described in full in \cite{rau:das:canfielddafilou:carillon:2019}.
%
\begin{figure*}
    \centering
    \includegraphics[width=2.25in]{Figures/ModePick_ex}
    \includegraphics[width=2.25in]{Figures/DecayFit_ex}
    \includegraphics[width=2.25in]{../Figures/HydroFlask/empty}
    \caption{\it{Modal analysis pipeline: (left) picking the mode frequencies,
    (center) estimating the decay rate of a single mode,
    (right) using a least-squares fit to estimate the complex
    amplitudes of the modes that ideally resynthesize the
    original signal.}}
    \label{fig:modal_analysis}
\end{figure*}
%
\subsubsection{Mode Synthesis} \label{sec:synthesis}
%
For synthesizing the modes we use the Max Matthews
phasor filter, as introduced in \cite{phasorfilter}.
This filter is described by the difference equation:
\begin{equation}
    y_m[n] = \alpha_m x[n] + e^{j\omega_m} e^{-1/\tau_m} y_m[n-1]
    \label{eq:phasor}
\end{equation}
%
where $\tau_m$ is the mode decay rate described above,
$\alpha_m$ is the complex amplitude of the mode, and $\omega_m$
is the mode frequency. This filter structure is known for
having favorable numerical properties, as well as for being
stable regardless of real-time parameter modulation.
\subsection{Water Level Analysis} \label{sec:water}
%
Next we examine the how the modal response of the water bottle
changes as the water level in the water bottle changes. 
%
\subsubsection{Frequency Variation} \label{sec:water-freq}
%
Measurements of the HydroFlask bottle show that as the water
level increases, the first mode frequency increases, while the
higher modes stay at the same frequency. This makes physical
sense since the lowest mode frequency corresponds to the Helmholtz
resonance of the bottle, and changing the water level
effectively decreases the size of the air column inside the
bottle, thereby causing the lowest mode frequency to increase
(@TODO: ask Mark if this is correct).
\newline\newline
@TODO: plot from Matlab (I think Mark has it \dots)
\hl{Yes, I have it somewhere. I'll find it soon. Should we mention the scanning measurements and show some? Can use it to show that the first mode is the "bending mode that depends on the mass". }\hl{\textbf{yes.}}
\newline\newline
We can use a cubic spline to model the movement of the first
mode frequency as the water level changes continuously, as
shown in \cref{fig:water-mode-freq}.
%
\begin{figure}[!htb]
    \centering
    \includegraphics[width=3in]{../Figures/Water_Freq}
    \caption{\it{Variation of the first mode frequency of the HydroFlask
                with the amount of water in the bottle}}
    \label{fig:water-mode-freq}
\end{figure}
%
\subsubsection{Damping Variation} \label{sec:water-damp}
%
Further analysis shows that the damping of the lowest two modes
varies with water level as well. We can similarly model the
variation of the mode decay rates with water level, using a
quartic polynomial to fit the average of decay rates of the first
two modes (see \cref{fig:water-mode-damp}).
%
\begin{figure}[!htb]
    \centering
    \includegraphics[width=3in]{../Figures/Water_Damping}
    \caption{\it{Variation of the first two modes decay rates
                 with the amount of water in the HydroFlask}}
    \label{fig:water-mode-damp}
\end{figure}
%
\subsection{Sticker Analysis} \label{sec:sticker}
%
Initially, we compared two 32 oz. HydroFlask water bottles, one
with stickers, one without, and noted that they had different
timbres. We then proceeded to take measurements of the
bottle covered in varying amount of removable stickers. We found
that the mode frequencies remained mostly unchanged with the
addition of stickers, but that the mode dampings had noticeable
variations (see \cref{fig:sticker-mode-damp}).
%
\begin{figure}[!htb]
    \centering
    \includegraphics[width=3in]{../Figures/StickerDamping}
    \caption{\it{Variation of the first ten modes decay rates
                 with the amount of stickers on the HydroFlask}}
    \label{fig:sticker-mode-damp}
\end{figure}
%
\subsection{Swinging Vibrato} \label{sec:swing}
%
When a water bottle is struck in such a way to produce an acoustic
response, it often swings back and forth a little bit. This swinging
causes the water within the bottle to move with the swinging, thereby
causing the lowest mode frequency to oscillate. This oscillation
manifests itself perceptually as a sort of vibrato effect. In
order to model this ``swinging vibrato'' we use the following
steps:
\begin{enumerate}
    \item Measure (or estimate) the height of the bottle.
    \item Calculate the swinging frequency of the bottle.
    \item Synthesize an initial amplitude and damping factor
        for the swinging oscillations.
\end{enumerate}
%
\subsubsection{Bottle Height}
%
In cases where the height of the water bottle cannot be measured directly,
it is possible to estimate the height from the bottle's modal characteristics.
For a typical cylindrical water bottle, the second lowest mode frequency
corresponds to the bottle's resonance along it's vertical length. As such,
the bottle height can be estimated as:
\begin{equation}
    L = \frac{v_{sound}}{2 f_2}
    \label{eq:bottle-height}
\end{equation}
where $v_{sound} = 343$ m/s is the speed of sound in air, $f_2$ is the second
lowest mode frequency, and $L$ is the height of the bottle.
%
\subsubsection{Swinging Frequency}
%
The frequency of a pendulum can be derived from Newton's Laws as:
\begin{equation}
    f_{swing} = \frac{1}{2\pi} \sqrt{\frac{g}{L}}
    \label{eq:swing-freq}
\end{equation}
where $g = 9.8 \text{ m}/\text{s}^2$ is the acceleration due to gravity at the surface
of Earth.
%
\subsubsection{Swinging Amplitude and Damping}
%
The amplitude of a water bottle's swinging oscillations typically depends
on how hard the bottle is struck. As such, the amplitude of the swinging
vibrato should vary proportionally with the desired volume of the synthesized
water bottle strike, for example with the ``velocity'' of a MIDI note.
The damping of a water bottle's swinging can be highly dependent on the method
by which the water bottle is anchored. As such, the damping factor is left
for the reader to determine for their own specific use cases.

@TODO: make a plot for this section.
%
\subsection{Impact Analysis} \label{sec:impact}
%
In real-world situations, a water bottle is typically struck using a body part
(knee, elbow, knuckles, etc), or using a striker that can be easily found in
nature, for instance a stick. With the goal of being able to synthesize these
types of impacts, we used an accelerometer to measure the impacts of
several body parts, as well as several types of drumsticks on a water bottle.
\Cref{fig:impact-time,fig:impact-freq} shows the time and frequency domain
measurements of the various impact types.
\begin{figure}
    \centering
    \includegraphics[width=3in]{../Figures/Impacts_time}
    \caption{\it{Time domain measurements of various impact drivers.}}
    \label{fig:impact-time}
\end{figure}
\begin{figure}
    \centering
    \includegraphics[width=3in]{../Figures/Impacts_freq}
    \caption{\it{Frequency domain measurements of various impact drivers.}}
    \label{fig:impact-freq}
\end{figure}


\section{Results} \label{sec:results}
%
Some general results \dots
%
\subsection{Water bottle Synthesizer} \label{sec:synth}
%
To demonstrate the musical power of water bottle synthesis, we have
implemented an 8-voice modal water bottle synthesizer as an audio
plugin (VST/AU), using the JUCE/C++ framework\footnote{\url{https://github.com/weAreROLI/JUCE}}.
The synthesizer includes controls for the amount of water in the bottle,
the number and placement of stickers on the bottle, and the option to
strike the water bottle with a variety of objects. Currently, the synthesizer
implements our model of the 32 oz. Wide Mouth HydroFlask, but could easily
be adjusted to model any other water bottle with similar measurement data.
The source code for this synthesizer is publicly available on
GitHub\footnote{\url{https://github.com/jatinchowdhury18/modal-water bottles}}.
%
\section{Conclusion} \label{sec:conclusion}
%
In this paper, we have discussed the synthesis of water bottle acoustics
using modal signal processing techniques. We have described the processes
of making acoustic measurements of the water bottles, as well as performing
modal analysis, with specific considerations for the amount of water
contained in the bottle, as well as the stickers placed on the exterior
of the bottle. Finally, we have implemented our modal model of a 32 oz.
Wide Mouth HydroFlask bottle as a real-time synthesizer plugin.
\newline\newline
Future research concerns \dots

\section{Acknowledgements}
%
The authors would like to thank Kim Kawczinski for helping us to contact
the HydroFlask corporation.

%\newpage
\nocite{*}
\bibliographystyle{IEEEbib}
\bibliography{references}

\end{document}
